\documentclass[journal,12pt,twocolumn]{IEEEtran}

\usepackage{enumitem}
\usepackage{amsmath}
\usepackage{amssymb}
\usepackage{gensymb}
\usepackage{graphicx}
\usepackage{txfonts}         
\usepackage{listings}
\usepackage{lstautogobble}
\usepackage{mathtools}
\usepackage{bm}
\usepackage{hyperref}
\usepackage{polynom}
\usepackage{siunitx}
\usepackage{verbatim}

\newcommand{\solution}{\noindent \textbf{Solution: }}
\providecommand{\pr}[1]{\ensuremath{\Pr\left(#1\right)}}
\providecommand{\brak}[1]{\ensuremath{\left(#1\right)}}
\providecommand{\cbrak}[1]{\ensuremath{\left\{#1\right\}}}
\providecommand{\sbrak}[1]{\ensuremath{\left[#1\right]}}
\providecommand{\der}[1]{\mathrm{d} #1}
\providecommand{\gauss}[2]{\mathcal{N}\ensuremath{\left(#1,#2\right)}}
\providecommand{\mbf}{\mathbf}
\providecommand{\z}[1]{{\mathcal{Z}}\{#1\}}
\providecommand{\ztrans}{\overset{\mathcal{Z}}{ \rightleftharpoons}}

\providecommand{\parder}[2]{\frac{\partial}{\partial #2} \brak{#1}}

\let\StandardTheFigure\thefigure
\let\vec\mathbf

%\numberwithin{equation}{section}
\renewcommand{\thefigure}{\theenumi}
\renewcommand\thesection{\arabic{section}}

\newcommand{\myvec}[1]{\ensuremath{\begin{pmatrix}#1\end{pmatrix}}}
\newcommand{\mydet}[1]{\ensuremath{\begin{vmatrix}#1\end{vmatrix}}}
\newcommand{\define}{\stackrel{\triangle}{=}}

\DeclareMathOperator*{\argmin}{arg\,min}
\DeclareMathOperator*{\argmax}{arg\,max}

\makeatletter
\def\pld@CF@loop#1+{%
    \ifx\relax#1\else
        \begingroup
          \pld@AccuSetX11%
          \def\pld@frac{{}{}}\let\pld@symbols\@empty\let\pld@vars\@empty
          \pld@false
          #1%
          \let\pld@temp\@empty
          \pld@AccuIfOne{}{\pld@AccuGet\pld@temp
                            \edef\pld@temp{\noexpand\pld@R\pld@temp}}%
           \pld@if \pld@Extend\pld@temp{\expandafter\pld@F\pld@frac}\fi
           \expandafter\pld@CF@loop@\pld@symbols\relax\@empty
           \expandafter\pld@CF@loop@\pld@vars\relax\@empty
           \ifx\@empty\pld@temp
               \def\pld@temp{\pld@R11}%
           \fi
          \global\let\@gtempa\pld@temp
        \endgroup
        \ifx\@empty\@gtempa\else
            \pld@ExtendPoly\pld@tempoly\@gtempa
        \fi
        \expandafter\pld@CF@loop
    \fi}
\def\pld@CMAddToTempoly{%
    \pld@AccuGet\pld@temp\edef\pld@temp{\noexpand\pld@R\pld@temp}%
    \pld@CondenseMonomials\pld@false\pld@symbols
    \ifx\pld@symbols\@empty \else
        \pld@ExtendPoly\pld@temp\pld@symbols
    \fi
    \ifx\pld@temp\@empty \else
        \pld@if
            \expandafter\pld@IfSum\expandafter{\pld@temp}%
                {\expandafter\def\expandafter\pld@temp\expandafter
                    {\expandafter\pld@F\expandafter{\pld@temp}{}}}%
                {}%
        \fi
        \pld@ExtendPoly\pld@tempoly\pld@temp
        \pld@Extend\pld@tempoly{\pld@monom}%
    \fi}
\makeatother

\lstset {
	frame=single, 
	breaklines=true,
	columns=fullflexible,
	autogobble=true
}             
                               
\title{Assignment 2 \\ \Large EE3900: Linear Systems and Signal Processing \\ \large Indian Institute of Technology Hyderabad}
\author{Mukunda Reddy \\ \normalsize AI21BTECH11021  \\ \large Oppenheim and Schafer}


\begin{document}

	\maketitle
	
	\textbf{Problem 2.13 (b)} \\
       which of the following discrete-time signals
        are eigenfunctions of stable, 
        LTI discrete-time systems? 
        Given $3^{n}$
        \\
    \solution\\
    Consider a linear time invariant system 𝐻
 with impulse response h[n] operating on some space of infinite length discrete time signals. Recall that the output $H[x(n)]$ . \\
$H(x[n])$ of the system for a given input $h[n]*x[n]$
 is given by the discrete time convolution of the impulse response with the input
 
 \begin{align*}
 H[x(n)] &= \sum^{\infty} _{k = -\infty} h[k]x[n-k]  
 \end{align*}
 lets try $x(n) = e^{sn}$ Computing the output fot the input
 \begin{align*}
 H(e^{sn}) &= \sum^{\infty} _{k = -\infty} h[k]e^{s(n-k)} \\
 &= \sum^{\infty} _{k = -\infty} h[k]e^{sn} e^{-sk} \\
 &= e^{sn} \sum^{\infty} _{k = -\infty} h[k]e^{-sk} \\ 
 \end{align*}
 
 Therefore we have 
 \begin{align*}
 H(e^{sn}) &= \lambda_s e^{sn} \\
 \lambda_s &= \sum^{\infty} _{k = -\infty} h[k]e^{-sk} \\
 \end{align*}
So the given inputs are eigenfunctions  and the output is given as shown.\\
 

The output signal is given by 
\begin{align*}
 H(3^{n}) &= \lambda_s 3^{n} \\
 \lambda_s &= \sum^{\infty} _{k = -\infty} h[k]3^{-k} \\
 \end{align*}
 
 We can write $3^{n}$ as $e^{n\ln 3}$.Here $s = \ln 3$.Therefore the given input signal is eigenfunction of LTI system.\\

\end{document}
